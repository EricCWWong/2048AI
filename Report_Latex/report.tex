\documentclass[11point]{article}
% set the margins
\usepackage[left=25mm,right=25mm,top=20mm,bottom=20mm]{geometry}
% for displaying images
\usepackage{graphicx}
% for equation support, e.g. eqref
\usepackage{amsmath}
% for table support, e.g. toprule
\usepackage{booktabs}
% for multiple authors as a block
\usepackage{authblk}

% I redefine vectors to be bold
\renewcommand{\vec}[1]{\boldsymbol{\mathbf{#1}}}

% I can define new macros for things I use a lot
\newcommand{\datapoint}{\vec{x}}
\newcommand{\weights}{\vec{w}}

\DeclareMathOperator{\argmax}{argmax}


\begin{document}

\title{Learning 2048 by Reinforcement Learning}
\author[*]{Chun Wing Wong}
\author[**]{Asbin Rai}
\author[**]{Wentao Xu}
\author[**]{Xiaowen Man}
\affil[*]{Dept. of Physics \& Astronomy}
\affil[**]{Dept. of Information Studies}
\affil[ ]{University College London, WC1E 6BT}
\date{\today}

\maketitle

\begin{abstract}
2048 is a very popular decision base game on mobile devices. The game has very simple concept and very clear targets to achieve, which makes it an interesting case for an AI to learn through reinforcement learning. We have, therefore, decided to investigate this and multiple simulations are made. Unfortunately, we are not very convinced that our method has succeeded in training a competent AI to play the game. 
\end{abstract}

\section{Introduction}
2048 has become a very popular game in the past 10 years. It has a very simple concept which involves moving the tiles around trying to achieve higher scores. As a group of people who enjoyed the game before, we have decided to attempt training an AI to play this game through pure reinforcement learning.  Multiple methods have been attempted but the results prove that 2048 perhaps is a game too complicated, despite the simplicity in the game concept, to be learnt through pure reinforcement learning. According to some online resources, it is shown that deep reinforcement learning with a 3 layer neural network is required to achieve decent results \cite{dedieu2017deep}.
\\

In the following, we will first present some background information such as a detailed explanation of the game mechanics. Then, we will present the method we used during the training process in the following section. Results will be presented after methods followed by a discussion section which will explain why we think our method has failed. 


% sections for your report can be in subfiles
\section{Background}
\subsection{Game Rules}
The game is played on a 4x4 grid. Initially, there will be two tiles on the board with values "2" or "4".  The players will make moves "Left", "Right", "Up" and "Down". If the adjacent tile in the same direction of the move is the same, the two tiles will merge and form a new tile with values of the sum of the original value. Moves are not allowed if no tiles changes position after that invalid move. After each valid move, the a new tile of either "2" or "4" will be generated and randomly assigned to one of the empty tile. The goal of this game is to obtain a tile of value "2048" after a series of moves. When there are no longer valid moves available to the player, the game is terminated and the player loses the game.

\subsection{Reinforcement Learning}
Reinforcement learning is learning what actions to take in a certain situation  that will maximise our rewards. The learner is not told what actions to take, but instead must discover which actions yield the most reward by trying them \cite{sutton2018reinforcement}.  In our project, we will have 4 actions available for our agent, they are "Left", "Right", "Up" and "Down".  We will need a representation to present the current state of the board. This will be denoted as $\phi$ in the rest of the report. We have chosen the simplest approximation --- linear approximation --- where the q-value of the action  is related to the state linearly as shown below:
\begin{equation*}
q_{a} = \phi ^{T} w_{a}
\end{equation*}
In order to play the game better, we have to update the weight after each episode. In this project, multiple schemes such as Monte-carlo has been considered but we have eventually decided to stick with the q-learning scheme.
\\

Finally, after calculating the q-value, we choose to use epsilon-greedy policy to decide which action the agent takes. Furthermore,  in the second simulation, we have decided to adopt the decaying epsilon-greedy policy, where we allow the agent to explore in the beginning but slowly decreasing the freedom of exploration as we reach the later stages of the learning.

\section{Method}
\subsection{Game Mechanics in Simulations}
Originally, we have considered simplifying the game mechanics by having a specific starting state and have a very fixed way of spawning new tiles to reduce randomness. However, we believe that this will defeat the whole point of the game, which mainly involves making moves that will limit the damage of randomness could do to our game. Thus, we have decided to follow the exact game mechanics with no reduction and simplification in the rules. This means that we will always start with a random board with 2 tiles with values on each of the tiles "2" or "4". Then, after each valid move, a new tile with values "2" or "4" will be generated  on a random empty tile. We loose when no valid moves are allowed.

\subsection{Representation and Q-value Calculation}
In our experiment, we have tried 2 types of representation, which are the simple representation and the relationship representation. They will be explained as follows. To calculate the q-value, we will use the linear approximation approach for simplicity.

\subsubsection{Simple Representation}
This is the simplest type of representation of our game, which is just simply a vector of 16 elements with each element representing the values of the individual tiles. For example, if we look at figure \ref{fig:game_board}, the state will simply be:
\begin{equation*}
\phi = (0,0,2,4,0,0,4,8,0,2,16,32,0,2,2,16)^{T}
\end{equation*}
To calculate the q-value, we will use our state vector acting on the weight vector representing each action, which can be written as:
\begin{equation*}
q_{a} = \phi ^{T} w_{a}
\end{equation*}
where $q_{a}$ is the q-value of that action and $w_{a}$ is the weight of that action.

\begin{figure}
	\centering
	\includegraphics[width=2.0in]{2048_Screenshot}
	\caption{An example game board. This image is obtained from Wikipedia.}
	\label{fig:game_board}
\end{figure}

\subsubsection{Relationship Representation}
Reading Barto's "Reinforcement Learning" book, I have learnt that, to improve my learning, my state vector needs to have information involving interaction between the tiles. In addition to that, I have also decided to use the book's convention where the action dependency will be put in the state instead of the weight vector. This means our Linear Approximation equation can be re-written as:
\begin{equation*}
q_{a} = \phi(a)^{T} w
\end{equation*}
When we play the game, whenever we consider a horizontal move, we always compare the tiles horizontally.  This means the interaction considered will be horizontal. We will represent the interaction by the difference between 2 tiles. Our state vector for move left will then be written as:
\begin{equation*}
\phi(L) = (\mbox{values of all 16 tiles}, \mbox{horizontal differences between each tile of each row})^{T}
\end{equation*}
To distinguish move "L" and "R" the horizontal differences in $\phi(R)$ will have opposite sign to that in $\phi(L)$. $\phi(U)$ and $\phi(D)$ can be calculated with similar method except we will consider vertical differences instead.  

\subsection{Actions}
As explained before, there are 4 available actions for the agent to choose from. They are "Left", "Right", "Up" and "Down". Our agent will choose the action by calculating the q-value of each action. Since there are lots of possible traces in this game, we choose the action using the epsilon-greedy policy to allow exploration by the agent. The epsilon-greedy policy is:

\begin{equation}
\pi(s,a) = \begin{cases} 1-\epsilon+\frac{\epsilon}{|A|}, & a = argmax_{a'}(q(s, a'))\\ \frac{\epsilon}{|A|} & \mbox{otherwise} \end{cases} \
\end{equation}
where $|A|$ is the number of available actions the agent has. In our case, $|A|$ will be most likely to be 4. As mentioned above, there are a huge possibility of traces in this game. To ensure we don't waste time exploring invalid moves, we have decided to restrict our agent to only choose from valid moves. This means if "Left" is not a valid move, we will only have 3 actions available for the agent to choose from and $|A| = 3$ in this case instead assuming the other 3 moves are still valid. 
\\

From multiple sources, I have also seen that instead of using a pure epsilon-greedy policy, people tend to use a decaying epsilon-greedy policy. This allows exploration at the beginning but it decreases the freedom of exploration as we get to the later episodes of the learning process. We have implemented this by a rather naive step-wise decaying scheme.  

\subsection{Reward Schemes}
Multiple reward schemes has been attempted and they both have their merit and drawbacks. We started off with a simple reward scheme where it will reward the value of the greatest value on the board after a move. For example, if after a move we obtain a board state shown in figure \ref{fig:game_board}, we will get 32 points. 
\\

Another reward scheme we have attempted will be the merge-reward scheme, where the agent will be rewarded whenever we merge two tiles into one. The reward will be the value of the new tile. For example, if the initial board is that shown in figure \ref{fig:game_board} and we move right, the two "2"s in the bottom role will merge and give a new tile "4". This will give us 4 points.

\subsection{Learning}
We have mainly used the q-learning equation when updating the weight vector, which is:
% note this uses the align environment
\begin{align}
% \input and \weights in this equation are defined using a macro 
% see the \newcommand lines at the top
\Delta w_a
% & helps neatly align equations over multiple lines 
& = \alpha(R_{t+1} + \gamma \mbox{max}_{a}(S_{t+1} (a), w_{t+1}) - q(S_{t}; w_{t})  \nabla_{w} q(S_{t}; w_{t}))
\\ % line break
& =  \alpha(R_{t+1} + \gamma \mbox{max}_{a}(S_{t+1} (a), w_{t+1}) - q(S_{t}; w_{t})  S_{t}))
\notag % don't give this line an equation number
\end{align}
where we have taken $\alpha = 0.0000001$, $\gamma = 0.9$. 
\\

We used the decaying-epsilon-greedy policy as explained above with $\epsilon=0.1$ initially. Finally, we have decided to play the game with at least 1000 episodes.


\input{results}
\section{Discussion}
In this report, we have presented our attempt to learn playing the game 2048. We have started with the simple representation which is simply a vector with all the values of tiles on the board. We used the a linear value approximation and the weight is updated by the q-learning scheme. Unfortunately, after 1000 episode learning process, there's no significant improvement in the outcome of the learning. Our agent seems to struggle once we obtain a 128 tile. Increasing the number of episodes turns out to have very little impact and our agent is still barely better than a random agent. We have come up with a few possible explaination, which are:
\\
\begin{enumerate}
	\item Our representation does not provide enough information for our agent to make good decisions. When we play the game, we don't make moves only based on the values of each individual tiles, we make moves based on the relationships between adjacent tiles. Therefore, perhaps we have to present this peice of information to our agent.
	\item The number of states available is so large that by running 1000 episodes or even 10000 episodes, we still have not experienced most available states. Thus, our learning is still not very complete.
	\item Randomness of the game means that even if we choose the best move possible that reduces the impact of randomness,  it is still possible that a game ruining tile will spawn. This will give a wrong message to  the AI that it is a bad move. This might impact our learning.
\end{enumerate}


Point 2 and 3 are very much the restriction of the game and our current computation power, so we realistically can only improve through point 1. To do so, we have added extra terms in our state vector which  encodes the differences between adjacent tiles, which we called the relationship representation. This will hopefully provide more information for our agent to make better moves.
\\

With this new representation, there are no significant improvements in terms of final results. We are still mainly stuck once reaching 128. However, we can argue that there are some interesting signs of something learnt. When our agent plays the game, it seems to try putting the larger values in a corner, which is one of the more successful strategy for this game. There seems to be a slightly higher probability of getting 512 when comparing with the original representation, but we cannot conclude that it is an outcome of learning instead of pure luck due to our small sample size.
\\

Like before, we have again tried to conclude by finding some explanations. By improving our representation, we have successfully obtained some good signs of learning. This gives us evidence that perhaps this representation modification is a meaningful one. During this representation refinement procedure, we have encounter several interesting questions as follows:
\begin{itemize}
	\item What information does our agent actually need to achieve meaningful learning?
	\item The power of value function approximation is that we do not require the simulation to experience all the specific state. Instead by going through one state, our agent should know how to make a decision for another state that shares some similarities. How should we capture these similarities?
	\item Let's consider the board shown in figure \ref{fig:game_board}. Suppose we have another board where the values are simply those in figure \ref{fig:game_board} multiplied by 2. If we consider the values, these 2 boards are completely different. However, in terms of adjacent differences in log 2 scale, these 2 boards are completely the same. Should these 2 board be represented by 2 distinguishable state representation (former) or by the same state representation (latter)?
\end{itemize}
I believe our second attempt to improve the representation did not fully answer these questions. If we can find better answers to these questions, then perhaps we will have a further improved learning outcome.
\\ 

The failure to achieve higher score consistently then must also be due to point 2 and 3. In fact, we have strong belief that the random nature of the game is causing complication to the learning process, as making a good move does not necessarily guarantee a better outcome.
\\

In addition to that, from this simulation, we have started to doubt whether linear approximation is a good approximation scheme for our value function. We believe to achieve better score, we might have to use other types of approximation methods.
\\

Last but not least, from some other online sources such as \cite{dedieu2017deep}, it is evident that pure reinforcement learning might not be sufficient for this game. A mixture of various machine learning methods might be required to obtain much better results. In particular, \cite{dedieu2017deep} has used reinforcement learning with neural networks and seems to have achieved much convincing result.
\\

Despite the rather disappointing outcome, we definitely have further understood the limitations of reinforcement learning. In the case of grid-world for example, when number of states and actions are very limited, it is very obvious what representation and reward schemes we should choose. However, for more complicated cases with an enormous set of states based on various parameters, it becomes challenging to choose a meaningful representation. Randomness could add complication to the process and increase the "error" of our learning.


\section*{Declaration}
This document represents the group report submission from the named authors for the project assignment of module: Foundations of Machine Learning and Data Science (INST0060), 2018-19. In submitting this document, the authors certify that all submissions for this project are a fair representation of their own work and satisfy the UCL regulations on  plagiarism.

% to put references in
\bibliography{report}
% define the bibliography style
\bibliographystyle{plain}

\end{document}
